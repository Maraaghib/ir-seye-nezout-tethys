
%% bare_jrnl_compsoc.tex
%% V1.4b
%% 2015/08/26
%% by Michael Shell
%% See:
%% http://www.michaelshell.org/
%% for current contact information.
%%
%% This is a skeleton file demonstrating the use of IEEEtran.cls
%% (requires IEEEtran.cls version 1.8b or later) with an IEEE
%% Computer Society journal paper.
%%
%% Support sites:
%% http://www.michaelshell.org/tex/ieeetran/
%% http://www.ctan.org/pkg/ieeetran
%% and
%% http://www.ieee.org/

%%*************************************************************************
%% Legal Notice:
%% This code is offered as-is without any warranty either expressed or
%% implied; without even the implied warranty of MERCHANTABILITY or
%% FITNESS FOR A PARTICULAR PURPOSE!
%% User assumes all risk.
%% In no event shall the IEEE or any contributor to this code be liable for
%% any damages or losses, including, but not limited to, incidental,
%% consequential, or any other damages, resulting from the use or misuse
%% of any information contained here.
%%
%% All comments are the opinions of their respective authors and are not
%% necessarily endorsed by the IEEE.
%%
%% This work is distributed under the LaTeX Project Public License (LPPL)
%% ( http://www.latex-project.org/ ) version 1.3, and may be freely used,
%% distributed and modified. A copy of the LPPL, version 1.3, is included
%% in the base LaTeX documentation of all distributions of LaTeX released
%% 2003/12/01 or later.
%% Retain all contribution notices and credits.
%% ** Modified files should be clearly indicated as such, including  **
%% ** renaming them and changing author support contact information. **
%%*************************************************************************


% *** Authors should verify (and, if needed, correct) their LaTeX system  ***
% *** with the testflow diagnostic prior to trusting their LaTeX platform ***
% *** with production work. The IEEE's font choices and paper sizes can   ***
% *** trigger bugs that do not appear when using other class files.       ***                          ***
% The testflow support page is at:
% http://www.michaelshell.org/tex/testflow/


\documentclass[10pt,journal,compsoc]{IEEEtran}
%
% If IEEEtran.cls has not been installed into the LaTeX system files,
% manually specify the path to it like:
% \documentclass[10pt,journal,compsoc]{../sty/IEEEtran}





% Some very useful LaTeX packages include:
% (uncomment the ones you want to load)


% *** MISC UTILITY PACKAGES ***
%
%\usepackage{ifpdf}
% Heiko Oberdiek's ifpdf.sty is very useful if you need conditional
% compilation based on whether the output is pdf or dvi.
% usage:
% \ifpdf
%   % pdf code
% \else
%   % dvi code
% \fi
% The latest version of ifpdf.sty can be obtained from:
% http://www.ctan.org/pkg/ifpdf
% Also, note that IEEEtran.cls V1.7 and later provides a builtin
% \ifCLASSINFOpdf conditional that works the same way.
% When switching from latex to pdflatex and vice-versa, the compiler may
% have to be run twice to clear warning/error messages.






% *** CITATION PACKAGES ***
%
\ifCLASSOPTIONcompsoc
  % IEEE Computer Society needs nocompress option
  % requires cite.sty v4.0 or later (November 2003)
  \usepackage[nocompress]{cite}
\else
  % normal IEEE
  \usepackage{cite}
\fi





% *** GRAPHICS RELATED PACKAGES ***
%
\ifCLASSINFOpdf
  % \usepackage[pdftex]{graphicx}
  % declare the path(s) where your graphic files are
  % \graphicspath{{../pdf/}{../jpeg/}}
  % and their extensions so you won't have to specify these with
  % every instance of \includegraphics
  % \DeclareGraphicsExtensions{.pdf,.jpeg,.png}
\else
  % or other class option (dvipsone, dvipdf, if not using dvips). graphicx
  % will default to the driver specified in the system graphics.cfg if no
  % driver is specified.
  % \usepackage[dvips]{graphicx}
  % declare the path(s) where your graphic files are
  % \graphicspath{{../eps/}}
  % and their extensions so you won't have to specify these with
  % every instance of \includegraphics
  % \DeclareGraphicsExtensions{.eps}
\fi
% graphicx was written by David Carlisle and Sebastian Rahtz. It is
% required if you want graphics, photos, etc. graphicx.sty is already
% installed on most LaTeX systems. The latest version and documentation
% can be obtained at:
% http://www.ctan.org/pkg/graphicx
% Another good source of documentation is "Using Imported Graphics in
% LaTeX2e" by Keith Reckdahl which can be found at:
% http://www.ctan.org/pkg/epslatex
%
% latex, and pdflatex in dvi mode, support graphics in encapsulated
% postscript (.eps) format. pdflatex in pdf mode supports graphics
% in .pdf, .jpeg, .png and .mps (metapost) formats. Users should ensure
% that all non-photo figures use a vector format (.eps, .pdf, .mps) and
% not a bitmapped formats (.jpeg, .png). The IEEE frowns on bitmapped formats
% which can result in "jaggedy"/blurry rendering of lines and letters as
% well as large increases in file sizes.
%
% You can find documentation about the pdfTeX application at:
% http://www.tug.org/applications/pdftex




\hyphenation{op-tical net-works semi-conduc-tor}
% simple interligne demandé par Mr Chaumette
\usepackage{setspace}
\doublespacing
\begin{document}
%
% paper title
% Titles are generally capitalized except for words such as a, an, and, as,
% at, but, by, for, in, nor, of, on, or, the, to and up, which are usually
% not capitalized unless they are the first or last word of the title.
% Linebreaks \\ can be used within to get better formatting as desired.
% Do not put math or special symbols in the title.
\title{Tethys's report : Collecting Sensor Data Without Infrastructure or Trust}

\author{Carlos NEZOUT,
        Serigne Amsatou SEYE}% <-this % stops a space}

\IEEEtitleabstractindextext{%
% Note that keywords are not normally used for peerreview papers.
\begin{IEEEkeywords}
Internet of Things, \ldots
\end{IEEEkeywords}}


% make the title area
\maketitle


\IEEEdisplaynontitleabstractindextext
\IEEEpeerreviewmaketitle



\section{Introduction}\label{sec:introduction}
The purpose of this report is to summarize a document published during 2018 IEEE/ACM Third International Conference on Internet-of-Things Design and Implementation. It's about Tethys, a system collecting sensor data without infrastructure. This paper will allow us to clear subjectively the relevance or not of this report.

First of all, we will contextualize the paper and present some related works raising existing approach on some meanfull terms of that subject. Then we'll clear goals of the paper, explain their contribution and some reviews before concluding.
\section{Context and Related Work}\label{sec:context}
\subsection{Context}
At the time of digital transformation, all innovative technologies are connected. Analysis and exploitation of the digital data that results from these devices provides axes of study to improve daily habits. In this case, this paper focuses on the consumption of water in residential buildings using Tethys. It's a "\emph{wireless water flow sensor that collects data at perfixture granularity without dependence on existing infrastructure and trusted gateways}". From a physical point of view, Tethys strongly decouples the energy dependence of the device to the infrastructure where it is deployed. It uses the water flow to mainly feed the proper functioning of its system and thus prevents any failures not correlated to infrastructure. Tethys uses crowdsourcing to collect all related data from residential consumers such as students. This process can present some inconvenient cause they can trust on consumer's phone reliability, so they need to enforce security between the sensing device and backend system ("\emph{reliability end-to-end}").

The first results they obtained show that this device is able to identify significant behaviors in the shower use. They placed about several sensors and collected data during 2 weeks. The analysis of these data makes it possible to sensitize a consumer punctually on his water consumption. This approach revealed a significant variation in average water consumption for a shower.
\subsection{Related Work}

\section{Paper Goals}\label{sec:paperGoals}
\section{Contribution}\label{sec:contribution}

\section{Discussion}\label{sec:discussion}
\subsection{Plus/Minus Reviews}


\section{Conclusion}\label{sec:conclusion}

\appendices
\section{Terms definition}
Here we define some terms that need to be explained.
\newline



\begin{thebibliography}{1}
\bibitem{IEEEhowto:}
H.~Chiang, J. ~Hong, K. ~Kiningham, L. ~Riliskis, P. ~Levis, and M.~Horowitz, \emph{Tethys: Collecting Sensor Data Without Infrasctructure or Trust}, \hskip 1em plus
  0.5em minus 0.4em\relax 2018 IEEE/ACM Third International Conference on Internet-of-Things Design and Implementation.
\end{thebibliography}

\end{document}
